\section{FAQ}
\subsection{Qu'est ce qu'un VPN décentralisé}
Un \texttt{VPN}, `Virtual Private Network`, réseau privé virtuel,
une extension des réseaux locaux et préserve la sécurité logique que
l'on peut avoir à l'intérieur d'un réseau local\footnote{\url{http://fr.wikipedia.org/wiki/Réseau\_privé\_virtuel}}.

\subsection{Que faire si je découvre un \texttt{bug}?}

Les sources du tNETacle sont disponibles \href{https://github.com/LaKabane/tNETacle}{ici}.
Sur la page, il est possible de créer un \href{https://github.com/LaKabane/tNETacle/issues/new}{ticket}
afin de signaler le problème.

Dans ce ticket, indiquez au minimum sur quel système d'exploitation, le problème rencontré et le moyen d'installation.
Essayez d'être le plus précis possible.

\subsection{Comment puis-je utiliser le tNETacle?}
Le tNETacle est fourni avec une interface graphique simple et concise.

Il vous faudra avant tout générer une clé afin de pouvoir communiquer avec les autres utilisateurs.
L'installation en fournira une. 

Dans la plupart des cas, la configuration donnée par défaut sera suffisante pour faire fonctionner le tNETacle.

Il vous suffira d'ajouter les personnes avec qui vous souhaitez communiquer. N'oubliez pas de partager votre clé publique.

\textcolor{red}{Ne surtout pas donner votre clé privée}, une personne mal intentionnée pourrait usurper votre identité.
