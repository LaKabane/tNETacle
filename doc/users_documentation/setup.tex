\section{Setup}
\subsection{Unix-Like}
\subsubsection{Gnu/Linux - Debian-based}
\textcolor{red}{Does not work for the moment}

Download the package .deb, that you can find on the \href{http://eip.epitech.eu/2013/tnetacle/tnetacle.html}{official website} of the tNETacle.

Then, you can install it with :

\begin{lstlisting}
sudo dpkg -i tNETacle.deb
\end{lstlisting}

\subsubsection{OpenBSD}
\textcolor{red}{Does not work for the moment}

Download the package for OpenBSD on the \href{http://eip.epitech.eu/2013/tnetacle/tnetacle.html}{official website} of the tNETacle.

Then, install this package with \href{http://www.openbsd.org/faq/fr/faq15.html}{pkg\_add(1)}

\begin{lstlisting}
sudo pkg_add -Ui tNETacle.tgz
\end{lstlisting}

It is also possible to install thanks to the \texttt{Ports}.
You have to go to the Ports directory and to the tNETacle directory.
Then, compile the sources and install it with the command make.

\begin{lstlisting}
cd PORTSDIR
make
sudo make install
\end{lstlisting}

\subsubsection{NetBSD}
\textcolor{red}{Does not work for the moment}

To install the tNETacle under NetBSD, you have to add
the repository to your repository list.
Then, update your package list inside to \texttt{pkgin}.
Finaly, execute the \texttt{pkgin} command to install the tNETacle.

This is the different command to execute :

\begin{lstlisting}
echo "http://eip.epitech.eu/2013/tnetacle/" > /usr/pkg/etc/pkgin/repositories.conf
sudo pkgin update
sudo pkgin install tNETacle
\end{lstlisting}

It is possible to make the installation with the ports, consult the section "OpenBSD"

\subsection{Windows}
\textcolor{red}{does not work for the moment}
Execute the setup.exe
