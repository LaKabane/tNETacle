\section{FAQ}
\subsection{What is a decentralized VPN}
A virtual private network (VPN) is a network that uses primarily public telecommunication infrastructure, such as the Internet, to provide remote offices or traveling users an access to a central organizational network.\footnote{\url{http://en.wikipedia.org/wiki/Virtual\_private\_network}}.

\subsection{I found a bug!?}

You can find the tNETacle sources \href{https://github.com/LaKabane/tNETacle}{here}.
Create a \href{https://github.com/LaKabane/tNETacle/issues/new}{ticket}, and don't forget to provide a precise description of your problem, your system configuration, and how you have compiled, if you have changed the default compilation settings.

\subsection{How to use the tNETacle?}
tNETacle provides a simple and efficient GUI.

When you have installed it, you need to create your keys (private and public) which permit to talk with the other users.

Most of the time, the default configuration does not need to be modified to run the tNETacle.

You just have to add the public keys of your friends and send your public key to be allowed to create or join a network.

\textcolor{red}{Never give your private key}.
