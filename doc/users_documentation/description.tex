\section{Le \texttt{tNETacle}, qu'est-ce que c'est?}
\texttt{tNETacle} est un VPN, ou réseau privé virtuel, décentralisé.

Ce type de réseau vise à mettre en place une connexion sûre à travers un réseau public tel qu'Internet, tout en apparaissant comme un réseau local aux yeux des machines qui le composent.
Ce réseau est décentralisé, chaque connexion se fait en direct, sans utiliser d'intermédiaire, ce qui le rend résistant aux pannes.

\texttt{tNETacle} cible aussi bien les entreprises et les techniciens que les particuliers.
Il offre une interface intuitive permettant à tous de créer une connexion sécurisée entre plusieurs ordinateurs sans risquer de divulguer des informations personnelles à une entreprise servant d'intermédiaire ou à une personne mal intentionnée.

Sa simplicité n'handicape pas pour autant le réseau. Celui-ci se veut extrêmement souple, il propose des possibilités de configuration poussées aux utilisateurs avancés et est capable de s'adapter aux obstacles imposés par les réseaux domestiques (translation d'adresses).

\texttt{tNETacle} est capable de fonctionner sur des équipements et systèmes divers, tout en s'adaptant aux configurations les plus modestes.
