\section{Installation}
\subsection{Unix-Like}
\subsubsection{Gnu/Linux - Fondé sur Debian}
\textcolor{red}{Non fonctionnel pour le moment}

Un paquet .deb est présent sur le \href{http://eip.epitech.eu/2013/tnetacle/tnetacle.html}{site officiel} du tNETacle.

Une fois ce paquet obtenu, lancez la commande d'installation du paquet.

\begin{lstlisting}
sudo dpkg -i tNETacle.deb
\end{lstlisting}

\subsubsection{OpenBSD}
\textcolor{red}{Non fonctionnel pour le moment}

Pour installer le tNETacle sous OpenBSD, téléchargez le
paquet fait pour OpenBSD présent sur le \href{http://eip.epitech.eu/2013/tnetacle/tnetacle.html}{site officiel} du tNETacle.

Ensuite, installez ce paquet. Pour cela, référez vous à la documentation de
\href{http://www.openbsd.org/faq/fr/faq15.html}{pkg\_add(1)}

\begin{lstlisting}
sudo pkg_add -Ui tNETacle.tgz
\end{lstlisting}


Il est aussi possible de faire une installation à l'aide des \texttt{Ports}.
Pour cela, allez dans le répertoire contenant les ports.
Il vous faut ensuite compiler les sources et les installer à l'aide de la commande make.

\begin{lstlisting}
cd PORTSDIR
make
sudo make install
\end{lstlisting}

\subsubsection{NetBSD}
\textcolor{red}{Non fonctionnel pour le moment}

Pour installer le tNETacle sous NetBSD, il faut ajouter le dépôt
contenant les paquets du tNETacle à votre liste de dépôts.
Ensuite, il faut mettre à jour votre liste de paquets interne à \texttt{pkgin}.
Finalement, il faut exécuter une commande afin d'installer le tNETacle.
Voici les commandes à exécuter :

\begin{lstlisting}
echo "http://eip.epitech.eu/2013/tnetacle/" > /usr/pkg/etc/pkgin/repositories.conf
sudo pkgin update
sudo pkgin install tNETacle
\end{lstlisting}

Il est aussi possible de faire l'installation à l'aide des ports.
Pour cela, référez vous à la section "OpenBSD".

\subsection{Windows}
\textcolor{red}{Non fonctionnel pour le moment}
Exécutez l'installateur.
