\section{Share}
\subsection{Libctl}
Cette bibliothèque est utilisé pour configurer les différentes interfaces que
l'on va créer. Ces interfaces permettent le paramétrage et l'administration
du core, ou permettront d'obtenir des informations sur le système
(statistiques, réseau, ...).

\subsection{Libtt}
Cette bibliothèque permet de créer une interface virtuelle \texttt{tun}.
Cette interface réseau servira pour la connexion du réseau virtuel.

Cette bibliothèque permet d'éviter les problèmes de licence qui pourrait se poser.
Également, elle propose une solution de portabilité. Elle fera beaucoup d'appelle à des
fonctions bas niveau, ce qui entraîne généralement des problèmes de portabilités.

\subsection{Réseau}
