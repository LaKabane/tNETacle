\section{Externe}
\subsection{Compression}
\subsubsection{Dépendance}
Pour la compression, nous utilisons la bibliothèque \href{http://zlib.net/}{\texttt{zlib}}.
Cette bibliothèque a pour particularité d'être reconnue pour sa stabilité et sa fiabilité.
De plus, elle n'est pas soumise à diverses brevets.

\subsubsection{Interface}
L'interface que nous avons implémenté se compose de deux fonctions :
\begin{itemize}
\item tnt\_compress
\item tnt\_uncompress
\end{itemize}

\texttt{tnt\_compress} prend en paramètre une chaîne de caractère et retourne une chaîne de caractère compressé.
\texttt{tnt\_uncompress} prend en paramètre une chaîne de caractère compressé et retourne une chaîne de caractère décompressé.

\subsection{Chiffrement}
\subsubsection{Dépendance}
Pour le chiffrement, nous utilisons la bibliothèque \href{http://www.openssl.org/}{\texttt{OpenSSL}}.
De même que pour la \texttt{zlib}, elle est reconnue pour sa stabilité et sa fiabilité.


\subsection{Gestion des événements}
\subsubsection{Dépendance}
Pour la gestion des événements systèmes (réseaux, IPC, ...), nous utilisons bibliothèque \href{http://libevent.org/}{\texttt{Libevent}}.
Connue pour être stable et performante, la libevent 6 est utilisée
par de nombreux projets libres et commerciaux.
