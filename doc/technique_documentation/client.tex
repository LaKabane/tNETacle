\section{Client}
\subsection{Utilité}
Le client permet de faire la liaison entre le tNETacle-core et l'utilisateur.
Il communiquera avec le tNETacle-core à l'aide d'une connexion réseau,
comme le ferait deux tNETacle-core pour communiquer.
Dans la suite du document, le client désignera le client graphique.
Aucun client textuel officiel ne sera fournit.

\subsection{Language}
Le client est écrit en C++ afin de profiter de la rapidité du langage et de pouvoir avoir accés à la partie objet de celui-ci.
Il ne demande pas d'avoir accés à des ressources bas niveau, contrairement au tNETacle-core.
De plus, le tNETacle est écrit en C, nous utiliserons le C++ afin de rester dans une logique de programmation.

Nous utilisons la bibliothèque Qt pour la partie graphique.
Elle est utilisable facilement sur tous les systèmes d'exploitations et est richement fournit en documentation.

\subsection{architecture}
